% 1108.tex
% 研究型文章学习笔记模板
% 建议使用 XeLaTeX 编译(支持中文)
\documentclass[11pt,a4paper]{article}
\usepackage[margin=2.5cm]{geometry}
\usepackage{fontspec}
\usepackage{xeCJK}
\usepackage{amsmath,amssymb}
\usepackage{graphicx}
\usepackage{float}
\usepackage{caption}
\usepackage{booktabs}
\usepackage{enumitem}
\usepackage[colorlinks=true,linkcolor=blue,citecolor=blue,urlcolor=blue]{hyperref}
\usepackage{fancyhdr}
\usepackage{tcolorbox}
\usepackage{datetime}

% 字体(可按需修改)
\setmainfont{TeX Gyre Termes}
\setCJKmainfont{SimSun} % Windows 常见中文字体,视系统调整

% 页眉页脚
\pagestyle{fancy}
\fancyhf{}
\lhead{\small 研究笔记}
\rhead{\small \PaperMeta}
\cfoot{\small \thepage}

% 元数据命令(在文档开始处填写)
\newcommand{\PaperMeta}{}
\newcommand{\PaperTitle}{}
\newcommand{\PaperAuthors}{}
\newcommand{\PaperVenue}{}
\newcommand{\PaperYear}{}
\newcommand{\PaperLink}{}

% 强调框
\newtcolorbox{highlight}{colback=yellow!10,colframe=yellow!60!black,boxrule=0.5pt}

\begin{document}

% ————— 填写元数据 —————
\renewcommand{\PaperTitle}{Observation and Modulation of the Quantum Mpemba Effect on a Superconducting Quantum Processor}
\renewcommand{\PaperAuthors}{Yueshan Xu, Cai-Ping Fang, Bing-Jie Chen, Ming-Chuan Wang}
\renewcommand{\PaperVenue}{ArXiv}
\renewcommand{\PaperYear}{2025}
\renewcommand{\PaperLink}{http://arxiv.org/abs/2508.07707}
\renewcommand{\PaperMeta}{\PaperTitle\ --- \PaperYear}

\begin{center}
    {\LARGE \textbf{\PaperTitle}}\\[6pt]
    {\small \PaperAuthors \quad | \quad \PaperVenue \quad | \quad \PaperYear}\\
    {\small \url{\PaperLink}}
\end{center}

\tableofcontents
\vspace{6pt}
\hrule
\vspace{10pt}

% ————— 快速摘要 —————
\section*{快速摘要(TL;DR)}
\begin{enumerate}[leftmargin=*]
    \item 3--5 行总结核心思想、主要贡献、适用场景与效果。
    \item 一句话亮点:例如“提出了一个轻量级的...,在 X 数据集上将误差降低了 Y\%”。
\end{enumerate}

\section{背景介绍}
\begin{enumerate}[leftmargin=*]
    \item In non-equilibrium quantum many-body systems, the quantum Mpemba effect (QME) emerges as a counterintuitive phenomenon: systems exhibiting greater initial symmetry breaking restore symmetry faster than those with less.
    \item \textbf{Three major research fields}
        \begin{itemize}
        \item The Mpemba effect, originally observed as faster freezing of hotter water than colder water under identical conditions, represents a counterintuitive non-equilibrium phenomenon with debated mechanisms.
        \item In open quantum systems interacting with an external environment through Markovian and non-Markovian processes. It is dominated by classical fluctuations, resembling the classical Mpemba effect
        \item In isolated quantum systems governed by intrinsic quantum dynamics. It is driven by intrinsic quantum fluctuations. \textbf{In isolated quantum systems, the quantum Mpemba effect (QME) manifests as a remarkable phenomenon: subsystems with greater initial symmetry breaking restore symmetry faster under a symmetry-preserving Hamiltonian.} The type of system under the study:
        \begin{itemize}
            \item Quasiparticle framework for integrable systems that include 1D or 2D models.
            \item In chaotic systems using random and dualunitary circuits.
            \item Non-ergodic contexts, such as many-body localized (MBL) systems
            \end{itemize}
        \end{itemize}
    \item 伪代码/核心算法要点
\end{enumerate}


\section{实验方案}
\begin{enumerate}
    \item Here, we report the observation and control of QME using a superconducting processor featuring a unique fully connected, tunable-coupling architecture that enables precise modulation from short- to long-range interactions.
    \item 调控参量
        \begin{enumerate}
            \item interaction range
            \item potential engineering 
            \item initial state selection
        \begin{enumerate}
    \item 方法类别(模型/算法/理论证明/系统实现)
    \item 主要结果/指标
    \item 可复现性:代码/数据/预训练模型(有/无/部分)
\end{enumerate}


\section{结论}
\begin{enumerate}
    \item In strong \textbf{short-range} coupling regimes, EA crossovers during quenches from \textbf{tilted Néel states} confirm the presence of QME.
    \item In \textbf{intermediate coupling regimes}, \textcolor{red}{synchronized EA and entanglement entropy} dynamics reveal the \textbf{suppression} of QME.
    \item QME reemerges with the introduction of \textcolor{red}{on-site linear potentials or quenches from tilted ferromagnetic states}, the latter proving robust against on-site disorder.
\end{enumerate}


\section{关键公式与推导}
Entanglement asymmetry (EA), defined as the relative entropy
\begin{equation}
\Delta S_A(t)=S\left(\rho_{A, Q}(t)\right)-S\left(\rho_A(t)\right)
\end{equation}
where $\rho_A$ is the reduced density matrix of subsystem $A$, $\rho_{A, Q}=\sum_q \Pi_q \rho_A \Pi_q$ denotes its projection onto the \textcolor{red}{conserved charge $Q_A$ eigenspaces}, and $S\left(\rho_A\right)$ is the von Neumann entropy of $\rho_A$. 

$\Delta S_A(t)$ captures the distance of $\rho_A$ from a \textcolor{red}{symmetric state $\rho_A,Q$} including contributions from non-local correlations within subsystem A that violate the symmetry.

This makes it a natural \textcolor{blue}{order parameter} for non-equilibrium dynamics, offering a fresh perspective on thermalization compared to traditional metrics like entanglement entropy.


% ————— 实验设置 —————
\section{实验与结果}
\begin{enumerate}
    \item 数据集、评价指标、基线方法
    \item 主要实验表格或图(可插入图片)
\end{enumerate}

\begin{figure}[H]
    \centering
    % \includegraphics[width=0.6\textwidth]{figs/result.png}
    \caption{示例:结果对比图(替换为真实图片)}
\end{figure}

% ————— 优缺点与评价 —————
\section{优点与局限}
\begin{enumerate}
    \item 优点:列出 3-5 点
    \item 局限:理论/实践/资源/泛化等问题
    \item 可信度评估:是否做了消融、显著性检验、多个 seed 等
\end{enumerate}

% ————— 思路与扩展 —————
\section{灵感与后续方向}
\begin{enumerate}
    \item 可借鉴的技巧与工具
    \item 可尝试的改进点
    \item 与自己研究的结合点(短期/中期)
\end{enumerate}

% ————— 实现笔记 —————
\section{实现/复现笔记}
\begin{enumerate}
    \item 关键超参表
    \item 训练资源与耗时
    \item 遇到的问题与解决方案(环境、数据预处理、模型不收敛等)
\end{enumerate}

% ————— 复现检查表 —————
\section{复现检查表}
\begin{enumerate}
    \item [\(\square\)] 数据集下载与预处理
    \item [\(\square\)] 代码实现(模型/训练/评估)
    \item [\(\square\)] 与论文结果对齐
    \item [\(\square\)] 随机种子与多次实验
\end{enumerate}

% ————— 重要引用 —————
\section{参考与延伸阅读}
% 使用 BibTeX 时在同目录下放 refs.bib 并取消注释下面两行
% \bibliographystyle{apalike}
% \bibliography{refs}
列出文章引用或推荐的延伸阅读(手动列出或使用 BibTeX)。

% ————— 个人笔记与 TODO —————
\section{个人笔记 / TODO}
\begin{enumerate}
    \item 待实现模块:
    \item 待阅读参考文献:
    \item 需要讨论的问题:
\end{enumerate}

\vfill
{\small 记录时间:\today\ \currenttime}

\end{document}